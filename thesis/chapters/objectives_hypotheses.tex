\chapter{Research Objectives and Hypotheses}

\section{Research Objectives}

\noindent
The primary objective of this research is to develop and validate hyperspectral imaging (HSI) methods for crack-related \textit{detection/identification at the time of measurement} in table grapes ('Scarlotta' cultivar) under field conditions, spanning both early-stage (absent-to-minimal/incipient cracking) and late-stage (advanced, macroscopically visible cracking) acquisition conditions. The work bridges the gap between spectroscopic research on cracking mechanisms and practical, non-destructive agricultural screening by leveraging imaging, machine learning, and robust whole-image aggregation.


\section{Research Hypotheses}

\noindent
The research is guided by the following hypotheses, structured hierarchically to reflect a progression from physical feasibility to algorithmic performance, robustness, early detection capability, and spectral efficiency.

\subsubsection{Hypothesis 1: Spectral Distinguishability (Physical Basis)}
\label{hyp:spectral_distinguishability}

\noindent
This hypothesis examines whether a fundamental physical difference exists in the spectral reflectance properties of healthy and cracked grape tissues.

\begin{itemize}
    \item \textbf{$H_0$:} There is no statistically significant difference in the spectral signatures of healthy and cracked tissues in 'Scarlotta' grapes.
    \item \textbf{$H_1$:} Statistically significant differences exist in the spectral signatures of healthy versus cracked grape skins at specific wavelengths, providing a physical basis for subsequent algorithmic discrimination.
\end{itemize}


\subsubsection{Hypothesis 2: Pixel-Level Classification Feasibility}
\label{hyp:pixel_level_feasibility}

\noindent
This hypothesis evaluates whether machine learning algorithms can effectively learn the spectral differences identified in Hypothesis~1 under controlled, pixel-level conditions.

\begin{itemize}
    \item \textbf{$H_0$:} A machine learning model trained on hyperspectral data cannot achieve classification performance, as measured by standard metrics (e.g., F1-score, precision/recall, or MCC), significantly higher than a random or baseline classifier at the pixel level.
    \item \textbf{$H_1$:} The proposed machine learning framework can successfully classify individual pixels as \textit{Cracked} or \textit{Healthy} with high performance, effectively learning the underlying spectral decision boundary.
\end{itemize}

\subsubsection{Hypothesis 3: Whole-Image Detection and Robustness}
\label{hyp:whole_image_robustness}

\noindent
This hypothesis assesses the robustness of the trained models when applied to realistic whole-image hyperspectral scenes containing complex background elements not present in the clean pixel dataset.

\begin{itemize}
    \item \textbf{$H_0$:} The model fails to distinguish between the spectral signatures of cracks and complex background clutter (e.g., leaves, woody tissue, or plastic), resulting in a high false positive rate when applied to full hyperspectral images.
    \item \textbf{$H_1$:} The spectral signatures of cracks are sufficiently distinct from background clutter to allow the model to effectively suppress noise and detect cracks within full-image scenes with high precision.
\end{itemize}

\subsubsection{Hypothesis 4: Early Detection Capability}
\label{hyp:early_detection}

\noindent
This hypothesis investigates the ability of the whole-image framework to detect cracking at its onset, prior to the appearance of visually observable symptoms.

\begin{itemize}
    \item \textbf{$H_0$:} The spectral changes associated with the early onset of cracking are too subtle to be reliably detected, limiting the model's applicability to late-stage damage.
    \item \textbf{$H_1$:} The spectral model is sufficiently sensitive to detect early physiological changes and micro-cracks, enabling early-stage identification of clusters exhibiting crack-related spectral signatures at the time of measurement (even when macroscopic cracking is not yet clearly visible).
\end{itemize}

\subsubsection{Hypothesis 5: Spectral Efficiency and Operational Feasibility}
\label{hyp:spectral_efficiency}

\noindent
This hypothesis examines whether the detection capabilities demonstrated in Hypotheses~3 and~4 can be preserved when using a reduced subset of informative spectral bands.

\begin{itemize}
    \item \textbf{$H_0$:} Reducing the number of spectral bands results in a significant degradation in detection performance, either in whole-image scenarios or during early-stage detection, implying that the full hyperspectral spectrum is required.
    \item \textbf{$H_1$:} A carefully selected subset of informative wavelengths captures the key physiological markers of cracking, enabling robust detection in both whole-image and early-stage scenarios using cost-effective multispectral sensing configurations.
\end{itemize}

\section{Summary of Main Contributions}

\noindent
To address the objectives and hypotheses outlined above, this study developed an end-to-end hyperspectral imaging framework. As will be detailed in the subsequent chapters, the findings first confirm the physical distinguishability of cracks at the pixel level, where an optimized XGBoost classifier achieved near-perfect performance (PR--AUC $\approx$ 1.0). Crucially, the study successfully transitioned this capability to realistic, whole-image vineyard scenes containing complex background clutter. By employing strategic backward feature selection, the framework reduced the spectral dimensionality to an optimal subset of 30 wavelengths while achieving robust full-image detection performance. Specifically, the aggregated model yielded an F1-score of 0.81 under early-stage (pre-symptomatic) conditions and 0.84 under late-stage conditions. These results not only validate the research hypotheses but also demonstrate the operational feasibility of transitioning HSI into a proactive, field-ready agricultural early-warning tool.

\section{Thesis Structure}

\noindent
This thesis is organized into several interconnected chapters. The \textit{Literature Review} chapter provides a comprehensive overview of grape cracking mechanisms and relevant precision agriculture techniques. The \textit{Materials and Methods} chapter describes the experimental design, data acquisition procedures, hyperspectral imaging systems, and modeling approaches used in this study. The \textit{Results} chapter presents the experimental findings, including pixel-level classification, whole-image crack detection, and wavelength selection experiments. The \textit{Discussion} chapter interprets these results in the context of early crack detection, physiological relevance, and practical deployment considerations. Finally, the \textit{Conclusions and Future Work} chapter summarizes the main contributions of the study and outlines directions for future research.
