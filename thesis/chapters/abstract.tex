% --- Abstract ---
\begin{abstract}

\noindent
Grape cracking is a severe physiological disorder that leads to substantial economic losses in table-grape production by reducing marketable yield and increasing susceptibility to secondary infections such as sour rot. Early, non-destructive detection is therefore critical for enabling timely management interventions. This study investigates the feasibility of hyperspectral imaging (HSI; VNIR 400--1000~nm) combined with machine learning for crack detection under field conditions, considering both early- and late-stage acquisitions (initial/minimal vs.\ advanced cracking). All spectral analyses were restricted to 450--925~nm and employed per-spectrum Standard Normal Variate (SNV) normalization.

\medskip
\noindent
A multi-stage analytical framework was developed, including pixel-level classification, whole-image crack detection, and wavelength selection. Pixel-level spectral signatures of healthy and cracked berries were manually annotated and used to train an XGBoost classifier, achieving high discrimination performance (accuracy $\approx$ 97\%, AUC $\approx$ 0.99) and confirming the existence of a distinct spectral separation between tissue states.

\medskip
\noindent
Direct application of pixel-level predictions to field scenes resulted in background-induced misclassification. To address this limitation, a whole-image crack-detection pipeline was developed, aggregating pixel-level CRACK probabilities using spatial post-processing (thresholding, morphological filtering, and patch/blob-based decision rules). The system was evaluated using a spatially disjoint row-based calibration--test split and reported with emphasis on the Cracked class. On the held-out test row, the pipeline achieved F1\textsubscript{CRACK} = 0.704 under early-stage acquisition and F1\textsubscript{CRACK} = 0.818 under late-stage acquisition.

\medskip
\noindent
Wavelength selection experiments demonstrated that comparable detection performance can be retained using reduced spectral-band subsets, indicating the potential for compact sensing configurations suitable for field deployment. Overall, the results establish the feasibility of HSI-based crack detection under realistic vineyard conditions and provide a methodological foundation for operational, non-destructive monitoring tools.

\medskip
\noindent
Furthermore, rigorous feature selection successfully reduced the full hyperspectral cube to an optimal subset of 30 wavelengths that strongly align with key physiological markers. This substantial dimensionality reduction demonstrates the commercial feasibility of developing dedicated, low-cost multispectral sensors for high-throughput screening. Finally, the proposed spatial pipeline exhibited strong spatial generalization, maintaining high detection precision when evaluated on entirely unseen vineyard rows. This proves the system's robustness against localized environmental and illumination variances, transitioning the solution from a controlled laboratory proof-of-concept to a viable field-ready architecture.

\end{abstract}
